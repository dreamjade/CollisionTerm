%% ****** Start of file apstemplate.tex ****** %
%%
%%
%%   This file is part of the APS files in the REVTeX 4.2 distribution.
%%   Version 4.2a of REVTeX, January, 2015
%%
%%
%%   Copyright (c) 2015 The American Physical Society.
%%
%%   See the REVTeX 4 README file for restrictions and more information.
%%
%
% This is a template for producing manuscripts for use with REVTEX 4.2
% Copy this file to another name and then work on that file.
% That way, you always have this original template file to use.
%
% Group addresses by affiliation; use superscriptaddress for long
% author lists, or if there are many overlapping affiliations.
% For Phys. Rev. appearance, change preprint to twocolumn.
% Choose pra, prb, prc, prd, pre, prl, prstab, prstper, or rmp for journal
%  Add 'draft' option to mark overfull boxes with black boxes
%  Add 'showkeys' option to make keywords appear
%\documentclass[aps,prd,preprint,amsmath,amssymb,groupedaddress]{revtex4-2}
\documentclass[aps,prd,twocolumn,amsmath,amssymb,groupedaddress]{revtex4-2}
%\documentclass[aps,prl,preprint,superscriptaddress]{revtex4-2}
%\documentclass[aps,prl,reprint,groupedaddress]{revtex4-2}

% You should use BibTeX and apsrev.bst for references
% Choosing a journal automatically selects the correct APS
% BibTeX style file (bst file), so only uncomment the line
% below if necessary.
%\bibliographystyle{apsrev4-2}

\begin{document}

% Use the \preprint command to place your local institutional report
% number in the upper righthand corner of the title page in preprint mode.
% Multiple \preprint commands are allowed.
% Use the 'preprintnumbers' class option to override journal defaults
% to display numbers if necessary
%\preprint{}

%Title of paper
\title{Collision}

% repeat the \author .. \affiliation  etc. as needed
% \email, \thanks, \homepage, \altaffiliation all apply to the current
% author. Explanatory text should go in the []'s, actual e-mail
% address or url should go in the {}'s for \email and \homepage.
% Please use the appropriate macro foreach each type of information

% \affiliation command applies to all authors since the last
% \affiliation command. The \affiliation command should follow the
% other information
% \affiliation can be followed by \email, \homepage, \thanks as well.
\author{YU-CHIA LIN}
\email[]{yuchialin@unm.edu}
%\homepage[]{Your web page}
%\thanks{}
%\altaffiliation{}
\affiliation{Department of Physics \& Astronomy, University of New Mexicoo, Albuquerque, New Mexico 87131, USA}
\affiliation{Department of Astronomy, University of Arizona, Tucson, Arizona 85721, USA}
\affiliation{Department of Physics, University of Arizona, Tucson, Arizona 85721, USA}
%Collaboration name if desired (requires use of superscriptaddress
%option in \documentclass). \noaffiliation is required (may also be
%used with the \author command).
%\collaboration can be followed by \email, \homepage, \thanks as well.
%\collaboration{}
%\noaffiliation

\date{\today}

\begin{abstract}
% insert abstract here
We investigate the collision term effects in the neutrino flavor oscillation.
\end{abstract}

% insert suggested keywords - APS authors don't need to do this
%\keywords{}

%\maketitle must follow title, authors, abstract, and keywords
\maketitle

% body of paper here - Use proper section commands
% References should be done using the \cite, \ref, and \label commands
\section{\label{sec:intro} INTRODUCTION}

\subsection{}
\subsubsection{}

\section{\label{sec:model} MODEL}
In this work, we consider the two-flavor neutrino flavor oscillation between $\nu_e$ and $\nu_\tau$, where $\nu_\tau$ represents a linear combination of the physical $\mu$ and $\tau$ flavor neutrino. On the other hand, we only consider monochromatic neutrino to simplify the model.

Denoting the (anti-)neutrino flavor density matrix and the corresponding polarization vector as
\begin{equation}
	\rho = \begin{bmatrix}
		\rho_{11} ~~~ \rho_{12} \\
		\rho_{21} ~~~ \rho_{22}
	\end{bmatrix}
\end{equation}
($\bar{\rho}$)and $\textbf{P}$($\bar{\textbf{P}}$) respectively, or
\begin{eqnarray}
	\rho_{ex} &\equiv& \rho_{12} = \frac{1}{2}\left(P_x + i P_y\right) \\
	\bar{\rho}_{ex} &\equiv& \bar{\rho}_{12} = \frac{1}{2}\left(\bar{P}_x + i \bar{P}_y\right) 
\end{eqnarray}, where $P_i$($\bar{P}_i$) is the $i$ component of $P$($\bar{P}$). 

Considering the collision terms, we have equations of motion (EOM) for polarization vectors as follows:
\begin{widetext}
\begin{eqnarray} 
	\dot{\textbf{P}} &=& \omega \textbf{B} \times \textbf{P}+ \mu (\textbf{P}-\bar{\textbf{P}}) \times \textbf{P} - \Gamma^{CC}_+ \textbf{P}_T + \Gamma^{AE}_+ (\textbf{P}^{AE} - \textbf{P}) + \Gamma^{AE}_- (P^{AE}_0 -P_0)\textbf{z} \\
	\dot{\bar{\textbf{P}}} &=& -\omega \textbf{B} \times \bar{\textbf{P}}+ \mu (\textbf{P}-\bar{\textbf{P}}) \times \bar{\textbf{P}} - \bar{\Gamma}^{CC}_+ \bar{\textbf{P}}_T + \bar{\Gamma}^{AE}_+ (\bar{\textbf{P}}^{AE} - \bar{\textbf{P}}) + \bar{\Gamma}^{AE}_- (\bar{P}^{AE}_0 -\bar{P}_0)\textbf{z} \\
	\dot{P_0} &=& \Gamma^{AE}_+ (P^{AE}_0 -P_0) + \Gamma^{AE}_- (P^{AE}_z -P_z) \\
	\dot{\bar{P}_0} &=& \bar{\Gamma}^{AE}_+ (\bar{P}^{AE}_0 -\bar{P}_0) + \bar{\Gamma}^{AE}_- (\bar{P}^{AE}_z -\bar{P}_z)
\end{eqnarray}
\end{widetext}
We then choose the parameters

\begin{eqnarray}
	\begin{bmatrix}
		\omega \\ 
		\textbf{B} \\ 
		\mu \\ 
		\theta
	\end{bmatrix}
	= \begin{bmatrix}
		0.3 ~km^{-1} \\ -sin 2\theta~\textbf{e}_x^f + cos 2\theta~\textbf{e}_z^f \\ 3 \times 10^5 ~km^{-1} \\ 10^{-6}
	\end{bmatrix}
\end{eqnarray}
\begin{eqnarray}
	\begin{bmatrix}
		\Gamma^{CC}_+ \\ \Gamma^{AE}_+ \\ \Gamma^{AE}_-
	\end{bmatrix}
	= \begin{bmatrix}
		0.5/11.4 ~km^{-1} \\ 0.5/0.417 ~km^{-1} \\ 0.5/0.417 ~km^{-1}
	\end{bmatrix}
\end{eqnarray}

\begin{eqnarray}
	\begin{bmatrix}
		\bar{\Gamma}^{CC}_+ \\ \bar{\Gamma}^{AE}_+ \\ \bar{\Gamma}^{AE}_-
	\end{bmatrix}
	= \begin{bmatrix}
		0.5/37.2 ~km^{-1} \\ 0.5/4.36 ~km^{-1} \\ 0.5/4.36 ~km^{-1} 
	\end{bmatrix}
\end{eqnarray}

\begin{eqnarray}
	\begin{bmatrix}
		\textbf{P}^{AE} \\ P^{AE}_0 \\ \bar{\textbf{P}}^{AE} \\ \bar{P}^{AE}_0
	\end{bmatrix}
	= \begin{bmatrix}
		2 ~\textbf{e}_z^f\\ 4 \\ 1.5 ~\textbf{e}_z^f \\ 3.5
	\end{bmatrix}
\end{eqnarray}

and theinitial condition
\begin{eqnarray}
\begin{bmatrix}
	\textbf{P} \\ P_0 \\ \bar{\textbf{P}} \\ \bar{P}_0
\end{bmatrix}
= \begin{bmatrix}
	2 ~\textbf{e}_z^f\\ 4 \\ 1.5 ~\textbf{e}_z^f \\ 3.5
\end{bmatrix}
\end{eqnarray}

\section{\label{sec:results} Results}

\subsection{\label{subsec:linear} Linearlized EOM}
When $P_x, ~P_y \ll P_z \sim 1$, nonlinear terms could be ignored, we have
\begin{widetext}
\begin{eqnarray}
	i \partial_t \rho_{ex} = \omega ~sin2\theta P_{z}
	+\left[-\omega ~cos2\theta-\sqrt{2}G_F(n_{\bar{\nu_e}}-n_{\bar{\nu_x}})- i \Gamma \right]\rho_{ex} + \sqrt{2}G_F(n_{\nu_e}-n_{\nu_x}) \bar{\rho}_{ex}\\
	i \partial_t \bar{\rho}_{ex} = - \omega ~sin2\theta \bar{P}_{z} +\left[+\omega ~cos2\theta + \sqrt{2}G_F(n_{\nu_e}-n_{\nu_x}) - i \bar{\Gamma} \right]\bar{\rho}_{ex} -\sqrt{2}G_F(n_{\bar{\nu_e}}-n_{\bar{\nu_x}}) \rho_{ex}
\end{eqnarray}
\end{widetext}
,
where
\begin{equation}
\begin{cases}
	B \equiv (sin 2\theta,0 ,cos 2\theta)
	\\
	\rho_{ex} \equiv \rho_{21} = P_x+i P_y
	\\
	\bar{\rho}_{ex} \equiv \bar{\rho}_{21} = \bar{P}_x+i \bar{P}_y
	\\
	\Gamma \equiv \Gamma^{AE} + \Gamma^{CC}
	\\
	\bar{\Gamma} \equiv \bar{\Gamma}^{AE} + \bar{\Gamma}^{CC}
\end{cases}
\end{equation}

Let
\begin{equation}
\begin{cases}
	\mu_+ \equiv \left(\frac{S+D}{2}\right) \mu
	\\
	\mu_- \equiv \left(\frac{S-D}{2}\right) \mu
\end{cases}
\end{equation}
and
\begin{equation}
A = \begin{bmatrix}
	-\omega ~cos2\theta-\mu_--i\Gamma ~~~~~~~~~ \mu_+ ~~~~~~~~~~~~~~~~\\ ~~~~~~~~~ -\mu_- ~~~~~~~~ \omega ~cos2\theta+\mu_+-i\bar{\Gamma}
\end{bmatrix}
\end{equation}

, we have
\begin{equation}
	\label{equ:EOM_matrix}
	i \partial_t \begin{bmatrix}
		\rho_{ex} \\ \bar{\rho}_{ex}
	\end{bmatrix} =
	\begin{bmatrix}
		\omega ~sin2\theta \\ - \omega ~sin2\theta
	\end{bmatrix} + A \begin{bmatrix}
		\rho_{ex} \\ \bar{\rho}_{ex}
	\end{bmatrix}
\end{equation}

Notice that, $P_{z}$ and $\bar{P}_{z}$ are both close to 1 under this limitation condition.

\subsection{\label{subsec:analytic} Analytic Solution}
Assume
\begin{equation}
	\begin{cases}
		\rho_{ex}(t) = \rho^0_{ex} + Q e^{-i\Omega t}
		\\
		\bar{\rho}_{ex}(t) = \bar{\rho}^0_{ex} + \bar{Q} e^{-i\Omega t}
	\end{cases}
\end{equation}
, where $\rho^0_{ex}$ and $\bar{\rho}^0_{ex}$ satisfy
\begin{equation}
	0= \begin{bmatrix}
		\omega ~sin2\theta \\ - \omega ~sin2\theta
	\end{bmatrix}  + 
	A \begin{bmatrix} \rho^0_{ex} \\ \bar{\rho}^0_{ex}
	\end{bmatrix}
\end{equation}
Then, from equation \ref{equ:EOM_matrix}, we have
\begin{equation}
	\Omega 
	\begin{bmatrix}
		Q \\ \bar{Q}
	\end{bmatrix}
	= A
	\begin{bmatrix}
		Q \\ \bar{Q}
	\end{bmatrix}
\end{equation}

\section{\label{sec:disscussion} DISCUSSION}

\section{\label{sec:conclusion} CONCLUSIONS}

% If in two-column mode, this environment will change to single-column
% format so that long equations can be displayed. Use
% sparingly.
%\begin{widetext}
% put long equation here
%\end{widetext}

% figures should be put into the text as floats.
% Use the graphics or graphicx packages (distributed with LaTeX2e)
% and the \includegraphics macro defined in those packages.
% See the LaTeX Graphics Companion by Michel Goosens, Sebastian Rahtz,
% and Frank Mittelbach for instance.
%
% Here is an example of the general form of a figure:
% Fill in the caption in the braces of the \caption{} command. Put the label
% that you will use with \ref{} command in the braces of the \label{} command.
% Use the figure* environment if the figure should span across the
% entire page. There is no need to do explicit centering.

% \begin{figure}
% \includegraphics{}%
% \caption{\label{}}
% \end{figure}

% Surround figure environment with turnpage environment for landscape
% figure
% \begin{turnpage}
% \begin{figure}
% \includegraphics{}%
% \caption{\label{}}
% \end{figure}
% \end{turnpage}

% tables should appear as floats within the text
%
% Here is an example of the general form of a table:
% Fill in the caption in the braces of the \caption{} command. Put the label
% that you will use with \ref{} command in the braces of the \label{} command.
% Insert the column specifiers (l, r, c, d, etc.) in the empty braces of the
% \begin{tabular}{} command.
% The ruledtabular enviroment adds doubled rules to table and sets a
% reasonable default table settings.
% Use the table* environment to get a full-width table in two-column
% Add \usepackage{longtable} and the longtable (or longtable*}
% environment for nicely formatted long tables. Or use the the [H]
% placement option to break a long table (with less control than 
% in longtable).
% \begin{table}%[H] add [H] placement to break table across pages
% \caption{\label{}}
% \begin{ruledtabular}
% \begin{tabular}{}
% Lines of table here ending with \\
% \end{tabular}
% \end{ruledtabular}
% \end{table}

% Surround table environment with turnpage environment for landscape
% table
% \begin{turnpage}
% \begin{table}
% \caption{\label{}}
% \begin{ruledtabular}
% \begin{tabular}{}
% \end{tabular}
% \end{ruledtabular}
% \end{table}
% \end{turnpage}

% Specify following sections are appendices. Use \appendix* if there
% only one appendix.
%\appendix
%\section{}

% If you have acknowledgments, this puts in the proper section head.
%\begin{acknowledgments}
% put your acknowledgments here.
%\end{acknowledgments}

% Create the reference section using BibTeX:
\bibliography{basename of .bib file}

\end{document}
%
% ****** End of file apstemplate.tex ******

